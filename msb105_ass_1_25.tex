% Options for packages loaded elsewhere
\PassOptionsToPackage{unicode}{hyperref}
\PassOptionsToPackage{hyphens}{url}
\PassOptionsToPackage{dvipsnames,svgnames,x11names}{xcolor}
%
\documentclass[
  a4paper,
]{article}

\usepackage{amsmath,amssymb}
\usepackage{iftex}
\ifPDFTeX
  \usepackage[T1]{fontenc}
  \usepackage[utf8]{inputenc}
  \usepackage{textcomp} % provide euro and other symbols
\else % if luatex or xetex
  \usepackage{unicode-math}
  \defaultfontfeatures{Scale=MatchLowercase}
  \defaultfontfeatures[\rmfamily]{Ligatures=TeX,Scale=1}
\fi
\usepackage{lmodern}
\ifPDFTeX\else  
    % xetex/luatex font selection
\fi
% Use upquote if available, for straight quotes in verbatim environments
\IfFileExists{upquote.sty}{\usepackage{upquote}}{}
\IfFileExists{microtype.sty}{% use microtype if available
  \usepackage[]{microtype}
  \UseMicrotypeSet[protrusion]{basicmath} % disable protrusion for tt fonts
}{}
\makeatletter
\@ifundefined{KOMAClassName}{% if non-KOMA class
  \IfFileExists{parskip.sty}{%
    \usepackage{parskip}
  }{% else
    \setlength{\parindent}{0pt}
    \setlength{\parskip}{6pt plus 2pt minus 1pt}}
}{% if KOMA class
  \KOMAoptions{parskip=half}}
\makeatother
\usepackage{xcolor}
\setlength{\emergencystretch}{3em} % prevent overfull lines
\setcounter{secnumdepth}{5}
% Make \paragraph and \subparagraph free-standing
\makeatletter
\ifx\paragraph\undefined\else
  \let\oldparagraph\paragraph
  \renewcommand{\paragraph}{
    \@ifstar
      \xxxParagraphStar
      \xxxParagraphNoStar
  }
  \newcommand{\xxxParagraphStar}[1]{\oldparagraph*{#1}\mbox{}}
  \newcommand{\xxxParagraphNoStar}[1]{\oldparagraph{#1}\mbox{}}
\fi
\ifx\subparagraph\undefined\else
  \let\oldsubparagraph\subparagraph
  \renewcommand{\subparagraph}{
    \@ifstar
      \xxxSubParagraphStar
      \xxxSubParagraphNoStar
  }
  \newcommand{\xxxSubParagraphStar}[1]{\oldsubparagraph*{#1}\mbox{}}
  \newcommand{\xxxSubParagraphNoStar}[1]{\oldsubparagraph{#1}\mbox{}}
\fi
\makeatother


\providecommand{\tightlist}{%
  \setlength{\itemsep}{0pt}\setlength{\parskip}{0pt}}\usepackage{longtable,booktabs,array}
\usepackage{calc} % for calculating minipage widths
% Correct order of tables after \paragraph or \subparagraph
\usepackage{etoolbox}
\makeatletter
\patchcmd\longtable{\par}{\if@noskipsec\mbox{}\fi\par}{}{}
\makeatother
% Allow footnotes in longtable head/foot
\IfFileExists{footnotehyper.sty}{\usepackage{footnotehyper}}{\usepackage{footnote}}
\makesavenoteenv{longtable}
\usepackage{graphicx}
\makeatletter
\newsavebox\pandoc@box
\newcommand*\pandocbounded[1]{% scales image to fit in text height/width
  \sbox\pandoc@box{#1}%
  \Gscale@div\@tempa{\textheight}{\dimexpr\ht\pandoc@box+\dp\pandoc@box\relax}%
  \Gscale@div\@tempb{\linewidth}{\wd\pandoc@box}%
  \ifdim\@tempb\p@<\@tempa\p@\let\@tempa\@tempb\fi% select the smaller of both
  \ifdim\@tempa\p@<\p@\scalebox{\@tempa}{\usebox\pandoc@box}%
  \else\usebox{\pandoc@box}%
  \fi%
}
% Set default figure placement to htbp
\def\fps@figure{htbp}
\makeatother
% definitions for citeproc citations
\NewDocumentCommand\citeproctext{}{}
\NewDocumentCommand\citeproc{mm}{%
  \begingroup\def\citeproctext{#2}\cite{#1}\endgroup}
\makeatletter
 % allow citations to break across lines
 \let\@cite@ofmt\@firstofone
 % avoid brackets around text for \cite:
 \def\@biblabel#1{}
 \def\@cite#1#2{{#1\if@tempswa , #2\fi}}
\makeatother
\newlength{\cslhangindent}
\setlength{\cslhangindent}{1.5em}
\newlength{\csllabelwidth}
\setlength{\csllabelwidth}{3em}
\newenvironment{CSLReferences}[2] % #1 hanging-indent, #2 entry-spacing
 {\begin{list}{}{%
  \setlength{\itemindent}{0pt}
  \setlength{\leftmargin}{0pt}
  \setlength{\parsep}{0pt}
  % turn on hanging indent if param 1 is 1
  \ifodd #1
   \setlength{\leftmargin}{\cslhangindent}
   \setlength{\itemindent}{-1\cslhangindent}
  \fi
  % set entry spacing
  \setlength{\itemsep}{#2\baselineskip}}}
 {\end{list}}
\usepackage{calc}
\newcommand{\CSLBlock}[1]{\hfill\break\parbox[t]{\linewidth}{\strut\ignorespaces#1\strut}}
\newcommand{\CSLLeftMargin}[1]{\parbox[t]{\csllabelwidth}{\strut#1\strut}}
\newcommand{\CSLRightInline}[1]{\parbox[t]{\linewidth - \csllabelwidth}{\strut#1\strut}}
\newcommand{\CSLIndent}[1]{\hspace{\cslhangindent}#1}

\makeatletter
\@ifpackageloaded{caption}{}{\usepackage{caption}}
\AtBeginDocument{%
\ifdefined\contentsname
  \renewcommand*\contentsname{Table of contents}
\else
  \newcommand\contentsname{Table of contents}
\fi
\ifdefined\listfigurename
  \renewcommand*\listfigurename{List of Figures}
\else
  \newcommand\listfigurename{List of Figures}
\fi
\ifdefined\listtablename
  \renewcommand*\listtablename{List of Tables}
\else
  \newcommand\listtablename{List of Tables}
\fi
\ifdefined\figurename
  \renewcommand*\figurename{Figure}
\else
  \newcommand\figurename{Figure}
\fi
\ifdefined\tablename
  \renewcommand*\tablename{Table}
\else
  \newcommand\tablename{Table}
\fi
}
\@ifpackageloaded{float}{}{\usepackage{float}}
\floatstyle{ruled}
\@ifundefined{c@chapter}{\newfloat{codelisting}{h}{lop}}{\newfloat{codelisting}{h}{lop}[chapter]}
\floatname{codelisting}{Listing}
\newcommand*\listoflistings{\listof{codelisting}{List of Listings}}
\makeatother
\makeatletter
\makeatother
\makeatletter
\@ifpackageloaded{caption}{}{\usepackage{caption}}
\@ifpackageloaded{subcaption}{}{\usepackage{subcaption}}
\makeatother

\ifLuaTeX
\usepackage[bidi=basic]{babel}
\else
\usepackage[bidi=default]{babel}
\fi
\babelprovide[main,import]{british}
% get rid of language-specific shorthands (see #6817):
\let\LanguageShortHands\languageshorthands
\def\languageshorthands#1{}
\ifLuaTeX
  \usepackage[english]{selnolig} % disable illegal ligatures
\fi
\usepackage{bookmark}

\IfFileExists{xurl.sty}{\usepackage{xurl}}{} % add URL line breaks if available
\urlstyle{same} % disable monospaced font for URLs
\hypersetup{
  pdftitle={Is reproducibility good enough?},
  pdfauthor={Amanda Skaar},
  pdflang={en-GB},
  colorlinks=true,
  linkcolor={blue},
  filecolor={Maroon},
  citecolor={Blue},
  urlcolor={Blue},
  pdfcreator={LaTeX via pandoc}}


\title{Is reproducibility good enough?}
\author{Amanda Skaar}
\date{Thursday 18 Sep, 2025}

\begin{document}
\maketitle
\begin{abstract}
A very short abstract. Put the abstract text here. One or two paragraphs
summarising what follows below.
\end{abstract}


\section{Introduction}\label{introduction}

What is this paper about? What is discussed? Why is it of any
consequence?

This paper is about reproducibility. First, it will resent literature
related to reproducibility in order to define what it is. In addition,
it will present articles that discuss some of the challenges. Then, the
theory on replicability is presented and what researchers think about
it. Furthermore, the theory is discussed, based in findings, including
whether reproducibility is asking too much. Is it good enough, or should
replicability become the new standard? It will also discuss possible
solutions to this. Finally, it will give a conclusion.

\section{Literature review}\label{literature-review}

Smart stuff from others about the topic.

Use a least 20 citations, a least 5 of them must be new (not from the
provided .bib file).

Use both in-line and normal citations.

Example:

Brase (2009) argues that bla bla bla. On the other hand it's claimed
that bla bla (Gentleman \& Lang, 2004; Knuth, 1984).

\subsection{Reproducibility}\label{reproducibility}

Reproducibility refers to the ability to use data from a study and
obtain the same results (Parashar et al. (2022)). This is an important
topic in research and academia because it relates to the validity of a
study.

The replication crisis, which is well-known in fields such as psychology
and medicine, has recently started to garner increased attention in
finance (Jensen et al., 2023). This phenomenon refers to the alarming
recognition that numerous financial studies and models, previously
regarded as reliable, do not yield consistent results when their methods
are applied to new datasets or evaluated under varying conditions.
Contributing factors to this problem include publication bias,
p-hacking, and the selective reporting of results(Hasso et al., 2025).

Several articles address the problems of generating reproducible data.
Some of the reasons are that it can be difficult to use the same
software, and raw data is not always available. Solutions to these
challenges have also been discussed. One such solution is a principle
called FAIR, which states that data should be Findable, Accessible,
Interoperable, and Reusable ((Parashar et al., 2022);(Turkyilmaz-van der
Velden et al., 2020)). Additionally, archives have been created for data
where researchers can upload data used in their studies. Furthermore,
there are requirements for data to be well-documented in order to be
uploaded to these archives.

\subsection{Replicability}\label{replicability}

In addition to reproducibility, there is also replicability. This
involves replicating a previous study using different data. It is
regarded as the gold standard ((Jasny et al., 2011)). At the same time,
expectations about replicability are more nuanced. If one does not
obtain the same results as a previous study, there can be many factors
at play ((National Academies of Sciences et al., 2019)).

National Academies of Sciences et al. (2019) mentions these factors: the
discovery of an unknown effect, inherent variability in the system,
inability to control complex variables, substandard research practices,
and, quite simply, chance.

A team of researchers tried to replicate articles published i en Journal
of Money, Credit and Banking. They were only able to replicate 2 of 54
articles. One thing they observed was that if the authors didnt
organized their data and code before submitting their articles, there
was little hope of reproducing the same results later ((McCullough et
al., 2008)).

\subsection{Quarto document}\label{quarto-document}

``Quarto documents are fully reproducible and support dozens of output
formats, like PDFs, Word files, presentations, and more'' (Grolemund \&
Wickham, n.d.).

Quarto documents are designed to be used in three different ways. One is
to communicating for decision-makers. A second way is to collaborate
with other scientists. The last is that it works as an environment in
which to do data science, as a lab notebook (Grolemund \& Wickham,
n.d.).

\section{Discussion of the research
question}\label{discussion-of-the-research-question}

\begin{itemize}
\tightlist
\item
  Should replicability be the norm or is this to much to ask for now?
\item
  Can Quarto documents help with reproducibility?
\item
  What problems remains and how can these be solved?
\end{itemize}

Let's take an example. A study aims to investigate whether richer people
are happier than poorer people. They survey a random sample of 1,000
individuals. Let's say that the study finds that richer people are
indeed happier than poorer people.

Now, a new study is to research exactly the same issue. If this study
does not arrive at the same conclusion, can we trust the previous study?
On the other hand, it's challenging to use the exact same program with
the same version that employs the same calculations. Is there sufficient
documentation for the first study? Is it defined adequately how
individuals define happiness? How is it determined whether someone is
rich or poor? These are also factors that make it difficult to replicate
the study.

This is a complex discussion, and there are several factors at play. It
is certainly an economic question and discovering that previous research
is accurate may not yield as much credit. At the same time, it is
important that research can be trusted.

\section{Conclusion}\label{conclusion}

\section{References}\label{references}

and

\begin{itemize}
\tightlist
\item
  Version number and reference to packages used
\item
  R version used
\end{itemize}

\phantomsection\label{refs}
\begin{CSLReferences}{1}{0}
\bibitem[\citeproctext]{ref-brase2009}
Brase, J. (2009). \emph{DataCite - a global registration agency for
research data}. 257261.

\bibitem[\citeproctext]{ref-gentleman2004}
Gentleman, R., \& Lang, D. T. (2004). Statistical {Analyses} and
{Reproducible Research}. \emph{Bioconductor Project Working Papers}.

\bibitem[\citeproctext]{ref-grolemund}
Grolemund, G., \& Wickham, H. (n.d.). \emph{R for data science}.

\bibitem[\citeproctext]{ref-hasso2025}
Hasso, T., Brosnan, M., Ali, S., \& Chai, D. (2025). Perceived problems,
causes, and solutions of finance research reproducibility and
replicability: A pre-registered report. \emph{Pacific-Basin Finance
Journal}, \emph{91}, 102564.
\url{https://doi.org/10.1016/j.pacfin.2024.102564}

\bibitem[\citeproctext]{ref-jasny2011}
Jasny, B. R., Chin, G., Chong, L., \& Vignieri, S. (2011). Again, and
again, and again. \emph{Science}, \emph{334}(6060), 12251225.

\bibitem[\citeproctext]{ref-jensen2023}
Jensen, T. I., Kelly, B., \& Pedersen, L. H. (2023). Is There a
Replication Crisis in Finance? \emph{The Journal of Finance},
\emph{78}(5), 2465--2518. \url{https://doi.org/10.1111/jofi.13249}

\bibitem[\citeproctext]{ref-knuth1984}
Knuth, D. E. (1984). Literate programming. \emph{The Computer Journal},
\emph{27}(2), 97111.

\bibitem[\citeproctext]{ref-mccullough2008}
McCullough, B. D., McGeary, K. A., \& Harrison, T. D. (2008). Do
economics journal archives promote replicable research? \emph{Canadian
Journal of Economics/Revue Canadienne d'économique}, \emph{41}(4),
14061420.

\bibitem[\citeproctext]{ref-nationalacademiesofsciences2019}
National Academies of Sciences, E., Affairs, P., Global, Committee on
Science, E., Information, B. on R. D., Sciences, D. on E. and, Physical,
Statistics, C. on A., Theoretical, Analytics, B. on M. S., Studies, D.
on E. and, Life, Board, N., Studies, R., Education, D. of B., Sciences,
S., Statistics, C. on N. and, Board on Behavioral, C., Science, C. on
R., \& in, R. (2019). \emph{Summary}. National Academies Press (US).
\url{https://www.ncbi.nlm.nih.gov/books/NBK547531/}

\bibitem[\citeproctext]{ref-parashar2022}
Parashar, M., Heroux, M. A., \& Stodden, V. (2022). Research
reproducibility. \emph{Computer}, \emph{55}(8), 16--18.
\url{https://doi.org/10.1109/MC.2022.3176988}

\bibitem[\citeproctext]{ref-turkyilmaz-vandervelden2020}
Turkyilmaz-van der Velden, Y., Dintzner, N., \& Teperek, M. (2020).
Reproducibility starts from you today. \emph{Patterns}, \emph{1}(6),
100099. \url{https://doi.org/10.1016/j.patter.2020.100099}

\end{CSLReferences}




\end{document}
